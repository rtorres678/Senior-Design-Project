Most instruments require serious skill and strength to use properly in order to get tones out of it. Whether it is a plucked instrument like a harp or a guitar, or even a violin, there are specific techniques on how it can be played. On top of it all, people with medical conditions or disabilities like carpal tunnel or arthritis cannot play most instruments due to the force required. On top of it all, in an increasingly digital world where everything is accessible at our finger tips, versatility is everything. Even, music these days is getting more and more complex, which often involves several instruments and relies on their syncopation to create a groove and melody that drives the song. Having some sort of instrument that is versatile, yet easy to use solves the key problem here. In essence, some type of MIDI device (at the core) provides us the greatest versatility, all the while keeping user inputs the same. Lasers come into play here as they can be used for detecting input, as there are two states: when the laser is blocked, or when the laser is not blocked. With those two states we can associate a note being triggered with when the laser is blocked. For people with medical disabilities like carpal tunnel and arthritis, an instrument with a laser based input is actually usable, as it requires minimal pressure and skill to trigger a note. Beyond this, a MIDI instrument that can also offer additional functionality like a synthesizer also tackles the problem of versatility, as most instruments and sounds at the core, are just a combination of sound waves with various characteristics. The synthesizer can model most instruments in the form of presets. Currently there isn’t any devices that use laser-based inputs, while also functioning as a MIDI device with audio produced by an internal synthesizer, which is why our project is unique because we are incorporating different things to create a device that allows a new platform of accessibility in music. 
