We will need a 3D printer to print a part of our frame. This will require us to use either the 3D printer in the Central Library, the 3D printer in Nedderman Hall Room 241, or the 3D printers in the Engineering Research Building. If later on we decide to use wood for some of our frame then we will need to borrow the laser cutter in Nedderman Hall Room 241. We will access to lab space when we assemble the laser harp. We will use lab space. Use of the laser cutter might be used to enhance the aesthetic of the laser harp. This will require us to borrow and use the laser cutter in Nedderman Hall Room 241.

To test the laser harp we can conduct tests in the space where we will assemble the laser harp. Since the laser harp is stationary it will not require additional space to operate as it does not move. Additionally this space will need a power outlet as the laser harp will need to be powered electrically. We will require a space to hold the frame and harp. Other than the space needed to hold the laser harp, we will not need to use other space.

We will require a soldering iron for wiring components of the Laser harp. Thus we will need to borrow this from the lab space. Additionally, because of the audio that our instrument will be broadcasting we will require an environment that does not require earplugs, which can muffle sound. Thus the makerspace in Nedderman Hall Room 241 will suffice.

We do not intend to lease any equipment, nor do we expect to purchase any additional equipment or machines that were not specified in for this project. All machines and equipment is available to be borrowed in either the makerspace in Nedderman 241 or the UTA Central Library FabLab.
