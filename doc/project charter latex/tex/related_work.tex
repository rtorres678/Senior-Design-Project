There are people who have previously created “laser harps” to model a real harp instrument for the purpose of demonstrating a harp with a more visually appealing element. Musicians and bands alike, such as Jean-Michel Jarre, Little Boots, Susumu Hirasawa, and many more have incorporated a laser synth in their live performances in the past. Additionally, people have used laser harps in public art installations at music festivals like Burning Man and Harmony Festival, and also in public art installations at various museums.

Furthermore, there are a few companies that are selling laser harp related products. PROLIGHT is one of them, which sells a controller, called the LH1 controller, that can turn any laser projector into a “frameless full color harp”, with added features that allow it to “be programmed to trigger any type of audio or video event, any visual image, sound or music, special effect or even pyrotechnics” \cite{PROLIGHT 2019}. It costs about 800 euros, which is roughly 880 USD, and is considered expensive. This type of product seems to be catered to live events, which explains the high cost. Another company called KROMALASER is selling an all-in-one laser harp device with various functionality for about 449 euros, which is roughly 490 USD. Although this is slightly cheaper and much more affordable than the product from PROLIGHT, this device seems very limited in its form factor, which is something we are trying to address with the design of our product.

Lastly, an optics design engineer named Jon Bumstead has built a few laser harps, which he shared online in the form of a guide on a popular DIY building site called Instructables. In his first iteration, he created a laser harp similar to the previous companies mentioned that has the lasers aimed upward meant to look like a “real harp”. This first iteration was purely a MIDI device as it required another device like a computer to read in the MIDI notes and play the appropriate sounds. In his second iteration, he built a “upright” laser harp which has a built-in MIDI player to play the audio right out of the device, and instead of the laser being aimed upwards, he implemented a design with “stacked laser beams that propagate horizontally” to then “reflect off mirrors to form square shaped beam paths”. In his words, “[w]ith this design, the lasers land on "frets," which makes it much simpler to block notes with a single finger”. Essentially improving the usage of the device \cite{Bumstead2019}. Lastly, he also gives the device functionality to select and play other instruments via a built-in rotary wheel. Ultimately, with Jon’s implementations we are still limited in the sounds/instruments we can play and we are constricted to the specific design he has chosen. Though, it is worth noting that we will be taking inspiration from his second iteration of the laser harp, especially given the internal hardware he utilizes to achieve a more versatile device.

Now overall, it would appear that a laser harp is the closest existing product to our project, as it uses lasers to trigger notes, as if it were a string on a real harp. We want to take this idea and propel it further by being able to play other instruments and sounds, with a versatile design and physical form factor that allow the user to play how they want. Specifically, we would like to implement a smaller harp-style design, and we see a few benefits to this layout: (1) people with carpal tunnel or arthritis can block a laser (note) without having to apply any pressure at all, and (2) the smaller size means the device will be more portable, and ultimately more accessible for use.  
