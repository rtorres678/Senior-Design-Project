%%% LaTeX Template: Article/Thesis/etc. with colored headings and special fonts
%%%
%%% Source: http://www.howtotex.com/
%%% Feel free to distribute this template, but please keep to referal to http://www.howtotex.com/ here.
%%% February 2011
%%%
%%% Modified January 2016 by CDM

%%%  Preamble
\documentclass[11pt,letterpaper]{article}
\usepackage[margin=1.0in]{geometry}
\usepackage[T1]{fontenc}
\usepackage[bitstream-charter]{mathdesign}
\usepackage[latin1]{inputenc}					
\usepackage{amsmath}						
\usepackage{xcolor}
\usepackage{cite}
\usepackage{hyphenat}
\usepackage{graphicx}
\usepackage{float}
\usepackage{subfigure}
\usepackage{sectsty}
\usepackage[compact]{titlesec} 
\usepackage[tablegrid]{vhistory}
\usepackage{pbox}
\usepackage{wrapfig}
\usepackage[export]{adjustbox}
\allsectionsfont{\color{accentcolor}\scshape\selectfont}

%%% Definitions
\definecolor{accentcolor}{rgb}{0.0,0.0,0.5} 
\newcommand{\teamname}{Resonance}
\newcommand{\productname}{Ltunes}
\newcommand{\coursename}{CSE 4317: Senior Design II}
\newcommand{\semester}{Fall 2019}
\newcommand{\docname}{Detailed Design Specification}
\newcommand{\department}{Department of Computer Science \& Engineering}
\newcommand{\university}{The University of Texas at Arlington}
\newcommand{\authors}{Amir Dhungana \\ Anish Yonjan \\ Roberto Torres \\ Raul Jimenez \\ Rabinson Shrestha \\ Nikhil Purohit}

%%% Headers and footers
\usepackage{fancyhdr}
	\pagestyle{fancy}						% Enabling the custom headers/footers
\usepackage{lastpage}	
	% Header (empty)
	\lhead{}
	\chead{}
	\rhead{}
	% Footer
	\lfoot{\footnotesize \teamname \ - \semester}
	\cfoot{}
	\rfoot{\footnotesize page \thepage\ of \pageref{LastPage}}	% "Page 1 of 2"
	\renewcommand{\headrulewidth}{0.0pt}
	\renewcommand{\footrulewidth}{0.4pt}

%%% Change the abstract environment
\usepackage[runin]{abstract}			% runin option for a run-in title
%\setlength\absleftindent{30pt}			% left margin
%\setlength\absrightindent{30pt}		% right margin
\abslabeldelim{\quad}	
\setlength{\abstitleskip}{-10pt}
\renewcommand{\abstractname}{}
\renewcommand{\abstracttextfont}{\color{accentcolor} \small \slshape}	% slanted text

%%% Start of the document
\begin{document}

%%% Cover sheet
{\centering \huge \color{accentcolor} \sc \textbf{\department \\ \university} \par}
\vspace{1 in}
{\centering \huge \color{accentcolor} \sc \textbf{\docname \\ \coursename \\ \semester} \par}
\vspace{0.5 in}
\begin{figure}[h!]
	\centering
   	\includegraphics[width=0.60\textwidth]{images/test_image}
\end{figure}
\vspace{0.5 in}
{\centering \huge \color{accentcolor} \sc \textbf{\teamname \\ \productname} \par}
\vspace{0.5 in}
{\centering \large \sc \textbf{\authors} \par}
\newpage


%\vspace{1 in}
%\centerline{January 13th, 2012}
%\newpage

%%% Revision History
\begin{versionhistory}
  	\vhEntry{0.1}{02.14.2020}{AD}{document creation}
  	\vhEntry{0.2}{02.20.2020}{AD}{complete draft}
  	\vhEntry{0.3}{02.21.2020}{AD}{release candidate 1}
  	\vhEntry{1.0}{02.21.2020}{AD}{official release}
  	\vhEntry{1.1}{05.11.2020}{AD}{revision and update}
  	
  
\end{versionhistory}
\newpage

%%% Table of contents
\setcounter{tocdepth}{2}
\tableofcontents
\newpage

%%% List of figures and tables (optional)
\listoffigures
\listoftables
\newpage

%%% Document sections
\section{Introduction}
 This section describes the purpose, use and intended user audience for the 'L-Tunes' laser harp instrument. L-Tunes is a laser-based digital instrument that uses laser signals and converts them to MIDI signals that trigger a musical note when the laser is blocked or obstructed. The device can be used to play sounds from different instruments like harp,  guitar, piano, etc. with presets and parameters to emulate other instruments and sounds. Users of L-Tunes will be able to play the device like an instrument and even create new sounds via built in sound generators. This product is intended for anyone, but particularly musicians, harp-enthusiasts, and children.

The L-Tunes laser harp should be used to play sounds like a harp, piano, etc. along with being versatile MIDI device than can be used to trigger sounds in a DAW (Digital Audio Workstation).

This device can be used by anyone,  however the device is intended to be used by musicians,  harp-enthusiasts, and young children. This device is designed for anyone looking for an alternative to a real harp with strings. 

\section{System Overview}
Laser Harp consists of four layers. The first layer is the power system, which is going to act as a power source for the device and consists of battery or similar power source. Second layer is frames and components which consists of laser beams, receptors, mister and speaker. Receptors are going to identify the interference in the laser beam and send the signal towards MIDI converter. Mister is being used for visibility whereas speakers are for audio output. Then, we have MIDI converter which is going to receive signals from the receptors and encode those signals as MIDI signals. Finally, the last layer is the sound module which is going to decode the MIDI signals, pass the resulting outcome to fluid synth which inturn sends the audio signals to speakers via soundcard. 

\begin{figure}[h!]
	\centering
 	\includegraphics[width=0.60\textwidth]{images/layers}
 \caption{Laser Harp Architectural Layer Diagram}
\end{figure}

\subsection{Layer X Description}
Each layer should be described separately in detail. Descriptions should include the features, functions, critical interfaces and interactions of the layer. The description should clearly define the services that the layer provides. Also include any conventions that your team will use in describing the structure: naming conventions for layers, subsystems, modules, and data flows; interface specifications; how layers and subsystems are defined; etc. 

\subsection{Layer Y Description}
Each layer should be described separately in detail. Descriptions should include the features, functions, critical interfaces and interactions of the layer. The description should clearly define the services that the layer provides. Also include any conventions that your team will use in describing the structure: naming conventions for layers, subsystems, modules, and data flows; interface specifications; how layers and subsystems are defined; etc. 

\subsection{Layer Z Description}
Each layer should be described separately in detail. Descriptions should include the features, functions, critical interfaces and interactions of the layer. The description should clearly define the services that the layer provides. Also include any conventions that your team will use in describing the structure: naming conventions for layers, subsystems, modules, and data flows; interface specifications; how layers and subsystems are defined; etc. 

\newpage
%\section{Subsystem Definitions \& Data Flow}
%There are 8 components in our laser harp instrument. These components are referred as subsystems which combine to give definition to the layers discussed above.

The Power System layer consists of single subsystem, Battery. It is main power source for all the components of the instrument. It will be comprised of single or combination of multiple 12V DC batteries. This subsystem will be place in the base of the instrument, will be wired to each component.

The Frames & Component system is comprised of three subsystems Lasers & Receptors, Mister and Speaker. Lasers & Receptors will be combination of multiple laser diodes positioned directly above the photoresistors, which will create laser beams. The interference of laser beams will be detected by the receptors and the resulting signals is sent to MIDI encoder. Mister acts as an individual unit and does not communicate with any other components whereas speakers get its input from the soundcard of the sound module layer.

The sound module layer contains three components. MIDI Decoders gets MIDI signals from the MIDI encoder. It interprets those signals, and which is fed into fluidsynth software. It allows the users to manipulate the frequency, depth, dampness, speed, etc. of the incoming signals. The resulting sound is in turn passed onto speakers for audio output via soundcard. 

The MIDI converter layers consists of single component i.e. MIDI encoder. The encoder is likely to be housed in a teensy microcontroller which takes the signals passed on by the receptors and changes it into MIDI signals. Those MIDI signals are forwarded towards MIDI decoder of sound module.

\begin{figure}[h!]
	\centering
 	\includegraphics[width=\textwidth]{images/data_flow}
 \caption{A simple data flow diagram}
\end{figure}

\newpage
\section{Power Layer Subsystems}
The details about the subsystem included in the power layer are discussed below. We are opting to go with traditional 12V 2A DC batteries, assuming that the batteries will be enough to power up the whole system. 

\subsection{Battery}
Battery is the only subsystem component included in this layer. They will act as the power source for all the components. They will be connected to other components via hookup wires and jumper cables.

\subsubsection{Assumptions}
The batteries will be able to provide enough power to run all the components of the instrument for a reasonable amount time without difficulties. All the batteries are new and have full capacity.

\subsubsection{Responsibilities}
It is responsible for providiing adequate power to all the components of the device.

\subsubsection{Subsystem Interfaces}
Input and output for the Battery subsystem are mentioned below.

\begin {table}[H]
\caption {Subsystem interfaces} 
\begin{center}
    \begin{tabular}{ | p{1cm} | p{6cm} | p{3cm} | p{3cm} |}
    \hline
    ID & Description & Inputs & Outputs \\ \hline
    \#xx & Battery  & \pbox{3cm}{12V DC batteries} & \pbox{3cm}{2A DC Power}  \\ \hline
    \end{tabular}
\end{center}
\end{table}

\newpage
\section{Frame and Component Layer Subsystems}
This layer is going to be the interaction layer between the users and the instrument. It consists of a frame possibly made from wood or 3D printed plastic parts. Lasers and photo-resistors will be lodged in to the cavities of the frame. A mister will be there to increase visibility. Traditional 5mm RGB tri-color LED will act as indicators for which lasers were interrupted. Amazon-Basics speakers with frequency range from 103 Hz - 20 KHz will be used for the output sound.

\subsection{Layer Hardware}
This layer will consist of traditional speakers,USB connected misters, LEDs,laser diodes and photo-resistors.

\subsection{Layer Operating System}
Teensy 3.6 will detect the any breakage in the circuit which will be encoded as MIDI signals.

\subsection{Layer Software Dependencies}
Basic C++ arduino script will be running on the teensy to detect any circuit changes.

\subsection{Lasers and receptors}
This consists of laser diodes and photo-resistors connected to a breadboard which is in turn connected to teensy micro controller which detects the lasers' interruption.

\begin{figure}[h!]
	\centering
 	\includegraphics[width=0.60\textwidth]{images/Frame.png}
 \caption{Lasers & Receptors subsystem description diagram}
\end{figure}

\subsubsection{Subsystem Hardware}
Laser diodes will be 6mm 5mW red dot laser head producing 650 nm laser waves running on 5V power supply whereas the photo-resistors are 5 mm GM5539 resistors with spectral peak of 540 nm.

\subsubsection{Subsystem Operating System}
N/A

\subsection{Mister}
It consists of an AGPtck aluminium mini mist maker which will be submerged in water to produce mists which will act as reflective medium for lasers to increase the visibility.

\begin{figure}[h!]
	\centering
 	\includegraphics[width=0.60\textwidth]{images/Mister.png}
 \caption{Mister subsystem description diagram}
\end{figure}

\subsubsection{Subsystem Hardware}
A homemade fog machine or a AGPtek mist maker of 1.8 inch diameter with DC 24V power source.

\subsection{Speakers}
They will act as the output of the whole instrument which consists of USB powered speakers for portability.

\begin{figure}[h!]
	\centering
 	\includegraphics[width=0.60\textwidth]{images/Speaker.png}
 \caption{Speakers subsystem description diagram}
\end{figure}

\subsubsection{Subsystem Hardware}
USB powered speaker(5V) with simple setup having easy front-access control for power and volume.










\newpage
\section{MIDI Layer Subsystems}
This layer is responsible for converting the analog signals coming from the layer and receptors subsystem to MIDI signals which can be analyzed by the midi decoder.

\subsection{Layer Hardware}
It consists of Teensy 3.6 micro-controller.

\subsection{Layer Operating System}
Teensy micro-controller is running a arduino script to detect the interference in the laser's circuit.

\subsection{Layer Software Dependencies}
MIDIUSB library is used in order to allow the micro-controller to encode the MIDI signals for the pi.

\subsection{MIDI Encoder}
It is C++ arduino script running on Teensy which converts the analog reading to MIDI signals.

\begin{figure}[h!]
	\centering
 	\includegraphics[width=0.60\textwidth]{images/MIDI.png}
 \caption{ MIDI encoder subsystem description diagram}
\end{figure}

\subsubsection{Subsystem Hardware}
Teensy 3.6 180 MHz micro-controller with 32 General purpose DMA channels.

\subsubsection{Subsystem Operating System}
Arduino 1.8 or similar in order correctly run the script.

\subsubsection{Subsystem Software Dependencies}
MIDIUSB library has been used to call functions to send notes or signals to raspberry pi.

\subsubsection{Subsystem Programming Languages}
Basic C++ arduino language is being is used for reading the signals from the circuit.





\newpage
\section{Sound Modules Layer Subsystems}
This layer is responsible for reading the MIDI signals and converting it into audio signals that is needed for the speakers to produce audio output. It consists of MIDI decoder, fluid synth and sound card which are integrated in a Raspberry Pi.

\subsection{Layer Hardware}
It consists of 1.5GHZ quad-core 64-bit raspberry pi.

\subsection{Layer Operating System}
It runs raspbian OS with kernel version 4.19.

\subsection{MIDI Decoder}
It is a driver program that acts as an bridge program between MIDI encoder and the fluidsynth.

\begin{figure}[h!]
	\centering
 	\includegraphics[width=0.60\textwidth]{images/decoder.png}
 \caption{MIDI decoder subsystem description diagram}
\end{figure}

\subsubsection{Subsystem Hardware}
It is running on Raspberry pi 4.

\subsubsection{Subsystem Operating System}
Raspbian OS is running a python 3.6 script that reads the encoding.

\subsubsection{Subsystem Programming Languages}
It uses Python 3.6.0 for server processing which uses socket library for communication with fluidsynth. 

\subsection{Fluidsynth}
Fluidsynth is a real-time MIDI synthesizer based on the SoundFont 2 specifications. 

\begin{figure}[h!]
	\centering
 	\includegraphics[width=0.60\textwidth]{images/fluidsynth.png}
 \caption{Fluidsynth subsystem description diagram}
\end{figure}

\subsubsection{Subsystem Operating System}
It is runs on startup on a raspbian OS.

\subsubsection{Subsystem Software Dependencies}
Fluidsynth is a library itself which reads in the MIDI input and plays according to the sound font files present. Moreover, socket library is called upon to connect with the fluidsynth process which in turn uses AF\_INET IPV4 protocol for communication.

\subsubsection{Subsystem Programming Languages}
Python is used to call upon the fluidsynth functions to control the gain, reverb and chorus of the audio output. 

\subsection{Soundcard}
Raspberry pi already comes in with a sound card integrated in its system. It is used for converting the output data from fluidsynth to audio data used by speakers for final output.

\begin{figure}[h!]
	\centering
 	\includegraphics[width=0.60\textwidth]{images/sound card.png}
 \caption{Sound Card subsystem description diagram}
\end{figure}

\subsubsection{Subsystem Hardware}
Raspberry pi 4 with compatible sound drivers for necessary sound processing.

\subsubsection{Subsystem Operating System}
Raspbian OS 









\newpage
\newpage
\section{User Interface}
This layer allows the user to control the sound font files to use such as church,acoustic guitar,etc. Also, it allows the user to change the reverb, gain and chorus of the sound as well as switch back and forth between the presets.

\subsection{Layer Hardware}
Elecrow 5 inch capacitive touch screen with 800*480 TFT LCD display is used as an interactive medium with the user.

\subsection{Layer Operating System}
It depends on the Raspbian OS.

\subsection{Layer Software Dependencies}
Python 3.6 is used to create the GUI for the application which uses tkinter library to create the button and sliders to control the audio output.  




\newpage

\section{Appendix A}
Final design model of the product,along with its views.
\begin{figure}[h!]
	\centering
 	\includegraphics[width=\textwidth, scale=0.2]{images/2.JPG}
 \caption{Final model of the final product}
\end{figure}
\begin{figure}[h!]
	\centering
 	\includegraphics[width=\textwidth]{images/4.JPG}
 \caption{Top view of the model}
\end{figure}
\begin{figure}[h!]
	\centering
 	\includegraphics[width=\textwidth]{images/3.JPG}
 \caption{Front view of the model}
\end{figure}

\newpage


%%% References
\bibliographystyle{plain}
\bibliographystyle{reference/IEEEtran_custom}
\bibliography{reference/refs}{}
  Fluidsynth manpage-https://manpages.debian.org/testing/fluidsynth/fluidsynth.1.en.html 

\\*Tkinter- http://effbot.org/tkinterbook/


\end{document}