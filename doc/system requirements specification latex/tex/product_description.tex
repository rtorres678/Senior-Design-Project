This section provides the reader with an overview of L-Tunes, a laser harp. The primary operational aspects of the product, from the perspective of end users, maintainers and administrators, are defined here. The key features and functions found in the product, as well as critical user interactions and user interfaces are described in detail.

\subsection{Features \& Functions}
The product generates sounds when a user obstructs the laser with their hand. The harp can also play other instruments by pressing the ‘>’ (arrow) button on the sound bank section. The harp can be used to play user generated sounds via the built in wave generators by first cycling through the sound bank, then by tinkering with the the ADSR sliders and wave cycle sliders to change the character of the sound that the two wave generators create in parallel.

The product does not function without a charged battery or power. The product does not play sounds beyond the sound bank.
The product primarily composed of [insert elements composing the product here].

The product looks like a smaller sized harp so that it is mobile (see Figure 1) [insert figure 1 here]. [specify the components here]. The product does not require any external elements how it can be used with a computer as a MIDI device, for example as an input source in a DAW (Digital Audio Workstation). 

\subsection{External Inputs \& Outputs}
No external data is required to flow into the device from external software for it to be used, as it has built in sounds and presets that can be selected by the user on the device itself. However, if the user chooses to use the product as a MIDI device then they need a computer with a DAW (Digital Audio Workstation) installed on it to be able to use the device as an input source for triggering audio playback. 
[insert table / diagram to show how to use the device as a MIDI device with an external computer/DAW] 

\subsection{Product Interfaces}
The device will feature a digital interface of sliders and buttons to control the device. The device will have a plug to power the device on and off, and a slider to control the volume, along with two sections of parameters for additional functionality. The first section will feature a screen that displays the instrument preset being used, with a button to change the preset. The second section will feature the audio synthesis portion of the device with (4) sliders to control the attack, decay, release and sustain of the sound and (4) buttons for the (2) wave generators that all one to cycle between sine, sawtooth, square, and triangle waves. 
[insert diagram of devices controls in the future] 
