%%% LaTeX Template: Article/Thesis/etc. with colored headings and special fonts
%%%
%%% Source: http://www.howtotex.com/
%%% Feel free to distribute this template, but please keep to referal to http://www.howtotex.com/ here.
%%% February 2011
%%%
%%% Modified January 2016 by CDM

%%%  Preamble
\documentclass[11pt,letterpaper]{article}
\usepackage[margin=1.0in]{geometry}
\usepackage[T1]{fontenc}
\usepackage[bitstream-charter]{mathdesign}
\usepackage[latin1]{inputenc}					
\usepackage{amsmath}						
\usepackage{xcolor}
\usepackage{cite}
\usepackage{hyphenat}
\usepackage{graphicx}
\usepackage{float}
\usepackage{subfigure}
\usepackage{sectsty}
\usepackage[compact]{titlesec} 
\usepackage[tablegrid]{vhistory}
\usepackage{pbox}
\allsectionsfont{\color{accentcolor}\scshape\selectfont}

%%% Definitions
\definecolor{accentcolor}{rgb}{0.0,0.0,0.5} 
\newcommand{\teamname}{Resonance}
\newcommand{\productname}{LTunes}
\newcommand{\coursename}{CSE 4316: Senior Design I}
\newcommand{\semester}{Fall 2019}
\newcommand{\docname}{Architectural Design Specification}
\newcommand{\department}{Department of Computer Science \& Engineering}
\newcommand{\university}{The University of Texas at Arlington}
\newcommand{\authors}{Amir Dhungana \\ Anish Yonjan \\ Roberto Torres \\ Raul Jimenez \\ Nikhil Purohit}

%%% Headers and footers
\usepackage{fancyhdr}
	\pagestyle{fancy}						% Enabling the custom headers/footers
\usepackage{lastpage}	
	% Header (empty)
	\lhead{}
	\chead{}
	\rhead{}
	% Footer
	\lfoot{\footnotesize \teamname \ - \semester}
	\cfoot{}
	\rfoot{\footnotesize page \thepage\ of \pageref{LastPage}}	% "Page 1 of 2"
	\renewcommand{\headrulewidth}{0.0pt}
	\renewcommand{\footrulewidth}{0.4pt}

%%% Change the abstract environment
\usepackage[runin]{abstract}			% runin option for a run-in title
%\setlength\absleftindent{30pt}			% left margin
%\setlength\absrightindent{30pt}		% right margin
\abslabeldelim{\quad}	
\setlength{\abstitleskip}{-10pt}
\renewcommand{\abstractname}{}
\renewcommand{\abstracttextfont}{\color{accentcolor} \small \slshape}	% slanted text

%%% Start of the document
\begin{document}

%%% Cover sheet
{\centering \huge \color{accentcolor} \sc \textbf{\department \\ \university} \par}
\vspace{1 in}
{\centering \huge \color{accentcolor} \sc \textbf{\docname \\ \coursename \\ \semester} \par}
\vspace{0.5 in}
\begin{figure}[h!]
	\centering
   	\includegraphics[width=0.60\textwidth]{images/test_image}
\end{figure}
\vspace{0.5 in}
{\centering \huge \color{accentcolor} \sc \textbf{\teamname \\ \productname} \par}
\vspace{0.5 in}
{\centering \large \sc \textbf{\authors} \par}
\newpage


%\vspace{1 in}
%\centerline{January 13th, 2012}
%\newpage

%%% Revision History
\begin{versionhistory}
  	\vhEntry{0.1}{11.02.2019}{AD}{document creation}
  	\vhEntry{0.2}{11.06.2019}{AY|AD}{complete draft}
  	\vhEntry{1.0}{11.08.2019}{AY|AD}{official release}
\end{versionhistory}
\newpage

%%% Table of contents
\setcounter{tocdepth}{2}
\tableofcontents
\newpage

%%% List of figures and tables (optional)
\listoffigures
\listoftables
\newpage

%%% Document sections
\section{Introduction}
 This section describes the purpose, use and intended user audience for the 'L-Tunes' laser harp instrument. L-Tunes is a laser-based digital instrument that uses laser signals and converts them to MIDI signals that trigger a musical note when the laser is blocked or obstructed. The device can be used to play sounds from different instruments like harp,  guitar, piano, etc. with presets and parameters to emulate other instruments and sounds. Users of L-Tunes will be able to play the device like an instrument and even create new sounds via built in sound generators. This product is intended for anyone, but particularly musicians, harp-enthusiasts, and children.

The L-Tunes laser harp should be used to play sounds like a harp, piano, etc. along with being versatile MIDI device than can be used to trigger sounds in a DAW (Digital Audio Workstation).

This device can be used by anyone,  however the device is intended to be used by musicians,  harp-enthusiasts, and young children. This device is designed for anyone looking for an alternative to a real harp with strings. 

\newpage
\section{System Overview}
Laser Harp consists of four layers. The first layer is the power system, which is going to act as a power source for the device and consists of battery or similar power source. Second layer is frames and components which consists of laser beams, receptors, mister and speaker. Receptors are going to identify the interference in the laser beam and send the signal towards MIDI converter. Mister is being used for visibility whereas speakers are for audio output. Then, we have MIDI converter which is going to receive signals from the receptors and encode those signals as MIDI signals. Finally, the last layer is the sound module which is going to decode the MIDI signals, pass the resulting outcome to fluid synth which inturn sends the audio signals to speakers via soundcard. 

\begin{figure}[h!]
	\centering
 	\includegraphics[width=0.60\textwidth]{images/layers}
 \caption{Laser Harp Architectural Layer Diagram}
\end{figure}

\subsection{Layer X Description}
Each layer should be described separately in detail. Descriptions should include the features, functions, critical interfaces and interactions of the layer. The description should clearly define the services that the layer provides. Also include any conventions that your team will use in describing the structure: naming conventions for layers, subsystems, modules, and data flows; interface specifications; how layers and subsystems are defined; etc. 

\subsection{Layer Y Description}
Each layer should be described separately in detail. Descriptions should include the features, functions, critical interfaces and interactions of the layer. The description should clearly define the services that the layer provides. Also include any conventions that your team will use in describing the structure: naming conventions for layers, subsystems, modules, and data flows; interface specifications; how layers and subsystems are defined; etc. 

\subsection{Layer Z Description}
Each layer should be described separately in detail. Descriptions should include the features, functions, critical interfaces and interactions of the layer. The description should clearly define the services that the layer provides. Also include any conventions that your team will use in describing the structure: naming conventions for layers, subsystems, modules, and data flows; interface specifications; how layers and subsystems are defined; etc. 

\newpage
\section{Subsystem Definitions \& Data Flow}
There are 8 components in our laser harp instrument. These components are referred as subsystems which combine to give definition to the layers discussed above.

The Power System layer consists of single subsystem, Battery. It is main power source for all the components of the instrument. It will be comprised of single or combination of multiple 12V DC batteries. This subsystem will be place in the base of the instrument, will be wired to each component.

The Frames & Component system is comprised of three subsystems Lasers & Receptors, Mister and Speaker. Lasers & Receptors will be combination of multiple laser diodes positioned directly above the photoresistors, which will create laser beams. The interference of laser beams will be detected by the receptors and the resulting signals is sent to MIDI encoder. Mister acts as an individual unit and does not communicate with any other components whereas speakers get its input from the soundcard of the sound module layer.

The sound module layer contains three components. MIDI Decoders gets MIDI signals from the MIDI encoder. It interprets those signals, and which is fed into fluidsynth software. It allows the users to manipulate the frequency, depth, dampness, speed, etc. of the incoming signals. The resulting sound is in turn passed onto speakers for audio output via soundcard. 

The MIDI converter layers consists of single component i.e. MIDI encoder. The encoder is likely to be housed in a teensy microcontroller which takes the signals passed on by the receptors and changes it into MIDI signals. Those MIDI signals are forwarded towards MIDI decoder of sound module.

\begin{figure}[h!]
	\centering
 	\includegraphics[width=\textwidth]{images/data_flow}
 \caption{A simple data flow diagram}
\end{figure}

\newpage
\section{Power System Layer Subsystems}
The details about the subsystem included in the power layer are discussed below. We are opting to go with traditional 12V 2A DC batteries, assuming that the batteries will be enough to power up the whole system. 

\subsection{Battery}
Battery is the only subsystem component included in this layer. They will act as the power source for all the components. They will be connected to other components via hookup wires and jumper cables.

\subsubsection{Assumptions}
The batteries will be able to provide enough power to run all the components of the instrument for a reasonable amount time without difficulties. All the batteries are new and have full capacity.

\subsubsection{Responsibilities}
It is responsible for providiing adequate power to all the components of the device.

\subsubsection{Subsystem Interfaces}
Input and output for the Battery subsystem are mentioned below.

\begin {table}[H]
\caption {Subsystem interfaces} 
\begin{center}
    \begin{tabular}{ | p{1cm} | p{6cm} | p{3cm} | p{3cm} |}
    \hline
    ID & Description & Inputs & Outputs \\ \hline
    \#xx & Battery  & \pbox{3cm}{12V DC batteries} & \pbox{3cm}{2A DC Power}  \\ \hline
    \end{tabular}
\end{center}
\end{table}

\newpage
\section{Frame and Component Layer Subsystems}
This layer is going to be the interaction layer between the users and the instrument. It consists of a frame possibly made from wood or 3D printed plastic parts. Lasers and photo-resistors will be lodged in to the cavities of the frame. A mister will be there to increase visibility. Traditional 5mm RGB tri-color LED will act as indicators for which lasers were interrupted. Amazon-Basics speakers with frequency range from 103 Hz - 20 KHz will be used for the output sound.

\subsection{Layer Hardware}
This layer will consist of traditional speakers,USB connected misters, LEDs,laser diodes and photo-resistors.

\subsection{Layer Operating System}
Teensy 3.6 will detect the any breakage in the circuit which will be encoded as MIDI signals.

\subsection{Layer Software Dependencies}
Basic C++ arduino script will be running on the teensy to detect any circuit changes.

\subsection{Lasers and receptors}
This consists of laser diodes and photo-resistors connected to a breadboard which is in turn connected to teensy micro controller which detects the lasers' interruption.

\begin{figure}[h!]
	\centering
 	\includegraphics[width=0.60\textwidth]{images/Frame.png}
 \caption{Lasers & Receptors subsystem description diagram}
\end{figure}

\subsubsection{Subsystem Hardware}
Laser diodes will be 6mm 5mW red dot laser head producing 650 nm laser waves running on 5V power supply whereas the photo-resistors are 5 mm GM5539 resistors with spectral peak of 540 nm.

\subsubsection{Subsystem Operating System}
N/A

\subsection{Mister}
It consists of an AGPtck aluminium mini mist maker which will be submerged in water to produce mists which will act as reflective medium for lasers to increase the visibility.

\begin{figure}[h!]
	\centering
 	\includegraphics[width=0.60\textwidth]{images/Mister.png}
 \caption{Mister subsystem description diagram}
\end{figure}

\subsubsection{Subsystem Hardware}
A homemade fog machine or a AGPtek mist maker of 1.8 inch diameter with DC 24V power source.

\subsection{Speakers}
They will act as the output of the whole instrument which consists of USB powered speakers for portability.

\begin{figure}[h!]
	\centering
 	\includegraphics[width=0.60\textwidth]{images/Speaker.png}
 \caption{Speakers subsystem description diagram}
\end{figure}

\subsubsection{Subsystem Hardware}
USB powered speaker(5V) with simple setup having easy front-access control for power and volume.










\newpage
\section{Sound Modules Layer Subsystems}
This layer is associated with reading the MIDI signals and converting it into audio signals required for the speaker to give out proper audio output. This consists of three subsystems MIDI decoder, fluidsynth and sound card.

\subsection{MIDI decoder}
MIDI decoder is a driver that gets the MIDI signals to the raspberry pi. It is a bridge software that transfers the encoded data for processing by fluidsynth. 
\begin{figure}[h!]
	\centering
 	\includegraphics[width=0.60\textwidth]{images/MIDI_decoder}
 \caption{MIDI decoder subsystem diagram}
\end{figure}

\subsubsection{Assumptions}
The drivers are compatible with the raspberry pi OS. 

\subsubsection{Responsibilities}
The drivers should successfully decode the MIDI data signals to be able to be processed by the fluidsynth software. There should be no loss of data.

\subsubsection{Subsystem Interfaces}
The MIDI decoder drivers are software drivers downloaded to the raspberry pi OS. It doesn't have any sub-system inside it


\subsection{FluidSynth}
FluidSynth is a real-time software synthesizer based on the SoundFont 2 specifications and has reached widespread distribution. It converts the MIDI signals to produce desired sound waves.

\begin{figure}[h!]
	\centering
 	\includegraphics[width=0.60\textwidth]{images/fluidsynthpng}
 \caption{FluidSynth subsystem diagram}
\end{figure}

\subsubsection{Assumptions}
The software already has default preset sounds built-in. It also has the functionality manipulate the sound to produce various sound effects. Also it is free, open source and easy to use.

\subsubsection{Responsibilities}
The software is used to convert MIDI signals to sound waves. It should have multiple different sound effects and create a wide range of sound octaves in different formats.

\subsubsection{Subsystem Interfaces}
The software is downloaded and installed into the raspberry pi. It is supported by the raspberry pi OS.


\subsection{Sound Card}
The sound card is integrated into the raspberry pi as drivers. It converts the data outputted by FluidSynth to sound data which is released through the speaker as the final output.

\begin{figure}[h!]
	\centering
 	\includegraphics[width=0.60\textwidth]{images/soundcard}
 \caption{Sound Card subsystem diagram}
\end{figure}

\subsubsection{Assumptions}
The sound card drivers are already built-in and integrated to the raspberry pi.

\subsubsection{Responsibilities}
The sound card should handle decent amount of data set. It should be able to produce a wide variety of sound data with different wavelengths.

\subsubsection{Subsystem Interfaces}
The sound card drivers are built-in software built-in to the raspberry pi OS. It doesn't have any sub-system inside it.





\newpage
\section{MIDI Layer Subsystems}
In this section, the layer is described in some detail in terms of its specific subsystems. Describe each of the layers and its subsystems in a separate chapter/major subsection of this document. The content of each subsystem description should be similar. Include in this section any special considerations and/or trade-offs considered for the approach you have chosen.

\subsection{MIDI encoder}
The MIDI encoder gets the signal from the laser receptors, converts it to MIDI data and sends to the Raspberry Pi for the output. This converts the laser receptor signals to MIDI data in order to convert it to sound data.  

\begin{figure}[h!]
	\centering
 	\includegraphics[width=0.60\textwidth]{images/subsystem}
 \caption{MIDI encoder subsystem description diagram}
\end{figure}

\subsubsection{Assumptions}
The MIDI encoder is connected to the Pi and the laser receptors via usb-A cable. It also has a built-in software that converts the data automatically.

\subsubsection{Responsibilities}
The laser recptors, when it detects an interference, sends signal to the MIDI encoder. Here, the encoder converts the signal data it received to a MIDI data so it can be maniplated to different sounds with the sound modules.

\subsubsection{Subsystem Interfaces}
Each of the inputs and outputs for the subsystem are defined here. Create a table with an entry for each labelled interface that connects to this subsystem. For each entry, describe any incoming and outgoing data elements will pass through this interface.

\begin {table}[H]
\caption {Subsystem interfaces} 
\begin{center}
    \begin{tabular}{ | p{1cm} | p{6cm} | p{3cm} | p{3cm} |}
    \hline
    ID & Description & Inputs & Outputs \\ \hline
    \#xx & Description of the interface/bus & \pbox{3cm}{input 1 \\ input 2} & \pbox{3cm}{output 1}  \\ \hline
    \#xx & Description of the interface/bus & \pbox{3cm}{N/A} & \pbox{3cm}{output 1}  \\ \hline
    \end{tabular}
\end{center}
\end{table}



\newpage

%%% References
\bibliographystyle{plain}
\bibliographystyle{reference/IEEEtran_custom}
\bibliography{reference/refs}{}

\end{document}
